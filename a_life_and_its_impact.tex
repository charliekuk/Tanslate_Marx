\chapter{生平与影响}
唯一可以与马克思的影响力相提并论的,恐怕只有像耶稣或是穆罕穆德这样的宗教人物了。在二十世纪下半叶的大部分时间里,地球上每十个人就有将近四个曾经在这样的政府统治下生活;这里的政府官员视自己为马克思主义者,并且无论多么难以置信,他们都宣称要用马克思主义原理来决定国家的运行方式。在这些国家中,马克思成为了某种世俗化的“基督”;他的著作成为了真理与权威的终极来源;他的画像则被恭敬地悬挂在各个地方。成千上万的人被马克思留下的遗产所深刻影响。

马克思的影响力并不局限于共产主义社会。保守政府通过引入社会改革来断绝马克思主义者革命性反对运动的根基。保守主义分子也用一种非良性的方法来应对:保守分子将极端民族主义视为对马克思主义威胁的回应,帮助墨索里尼和希特勒夺取到权力。甚至当没有内部革命的威胁时,国外马克思主义敌人的存在也会给政府增加军费开支、以国家安全之名限制个人权利这些行为提供合理性。