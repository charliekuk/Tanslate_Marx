\chapter{前言}
市面上有许多关于马克思的书,但是却很难找到一个优秀的简介来介绍他的思想。马克思就许多不同的主题撰写了浩如烟海的著作,以至于想要从整体上把握他的思想绝非易事。我相信总有一个核心观点,一个世界观,可以统一马克思的思想,并对那些令人困惑的特点给予解释。在这本书中,我试着向那些对马克思的作品只有很少或者压根没有任何了解的读者传递一个可以理解的中心观点。如果我已经成功了,那便不需要任何其他借口再向这些关于马克思与马克思主义的众多著作中多添一笔了。

关于马克思生平这一部分,我尤其要感谢David McLellan精彩的叙述:\textit{Karl Marx: His Life and Thought (Macmillan, London, 1973)}。我对马克思关于历史概念的理解深受G.A.Cohen所著\textit{Karl Marx’s Theory of History: A Defence (Oxford University Press, Oxford, 1979)}的影响,尽管我并不是完全接受这项颇具挑战性研究的全部结论。 Gerald Cohen 对这本书的初稿提出了详细的意见,帮助我纠正了一些错误。Robert Heibroner, Renata Singer以及Marilyn Weltz也给草稿做出了很多有益的评论,我对此表达深深的谢意。

为了确保文章清晰的表达,我会偶尔对引用到的马克思著作的翻译做一些微小的改动。

最后,如果没有这个系列的总编辑 Keith Thomas 和牛津大学出版社 Henry Hardy的邀请使我参与其中,我是无论如何都不会尝试去写这本书的;此外,如果没有 Monash University 允许我放假一段时间,我恐怕也不能完成这本小书。


\rightline{Peter Singer}

\rightline{华盛顿特区,1979年六月}
